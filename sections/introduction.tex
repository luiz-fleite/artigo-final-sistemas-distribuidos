% SEÇÃO 1: INTRODUÇÃO
\section{Introdução} 
\textcolor{red}{[MAX]}

A segurança da informação é um dos desafios mais críticos no projeto de Sistemas Distribuídos. Diferente de sistemas centralizados, onde o sistema operacional possui controle total sobre o acesso à memória e aos recursos, os sistemas distribuídos dependem de redes de comunicação para a troca de mensagens. Essas redes, frequentemente, são canais inseguros onde agentes mal-intencionados podem interceptar (sniffing), modificar ou falsear (spoofing) dados.

O problema central reside na autenticação: como um cliente pode provar sua identidade para um servidor remoto sem transmitir sua senha em texto claro pela rede? O envio de senhas, mesmo que esporádico, expõe as credenciais a ataques de interceptação. Além disso, em um ambiente com múltiplos serviços (arquivos, impressão, banco de dados), exigir que o usuário insira sua senha repetidamente degrada a usabilidade e aumenta o risco de comprometimento.

Neste contexto, o protocolo Kerberos surge como uma solução robusta, fundamentada no uso de criptografia simétrica e em uma entidade de terceira parte confiável. O objetivo deste trabalho é analisar a arquitetura do Kerberos, detalhando seu funcionamento, a troca de mensagens para obtenção de tickets e sua aplicação prática em ambientes corporativos modernos.
