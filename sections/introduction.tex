% SEÇÃO 1: INTRODUÇÃO
\section{Introdução}

A segurança da informação compõe um dos pilares fundamentais no projeto e implementação de Sistemas Distribuídos. Segundo Tanenbaum e Van Steen \cite{tanenbaum2017}, a abertura e a conectividade inerentes a esses sistemas, embora permitam o compartilhamento eficiente de recursos, introduzem vulnerabilidades críticas inexistentes em sistemas centralizados. Enquanto um sistema operacional local detém controle total sobre o acesso à memória e aos periféricos, os sistemas distribuídos dependem de redes de comunicação para a troca de mensagens. Frequentemente, essas redes comportam-se como canais inseguros, suscetíveis à interceptação passiva (\textit{sniffing}) e à modificação ativa ou mascaramento (\textit{spoofing}) de dados por agentes mal-intencionados.

O cerne dessa problemática reside na \textbf{autenticação}: a necessidade de um cliente provar sua identidade a um servidor remoto sem a exposição de sua senha em texto claro pela rede. O tráfego de credenciais por meios inseguros, ainda que ocorra de forma esporádica, vulnerabiliza o sistema a ataques de captura de tráfego. Adicionalmente, em cenários compostos por múltiplos serviços, como sistemas de arquivos e bancos de dados, a exigência de repetidas inserções de senhas degrada a usabilidade e induz o usuário a adotar práticas de segurança frágeis.

Neste cenário, o protocolo \textbf{Kerberos} consolida-se como uma solução padrão para autenticação em redes inseguras. Desenvolvido originalmente no Projeto Athena do MIT \cite{neuman1994}, o protocolo fundamenta-se no uso de criptografia simétrica e na existência de uma Terceira Parte Confiável (Trusted Third Party - TTP) para mediar a confiança entre clientes e servidores. O Kerberos permite que entidades comprovem sua identidade mutuamente, garantindo a integridade e a confidencialidade das sessões estabelecidas.

Este artigo tem como objetivo analisar a arquitetura de segurança do protocolo Kerberos. Serão detalhados os conceitos fundamentais de criptografia e canais seguros, o fluxo de troca de mensagens para obtenção de tickets e a mitigação de ataques de repetição. Por fim, apresenta-se um estudo de caso sobre sua implementação no Microsoft Active Directory, demonstrando sua aplicabilidade em infraestruturas modernas.