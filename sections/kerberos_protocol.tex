% SEÇÃO 3: O PROTOCOLO KERBEROS
\section{O Protocolo Kerberos}
\textcolor{red}{[LA]}

Desenvolvido pelo MIT no projeto Athena, o Kerberos é um protocolo de autenticação de rede projetado para fornecer autenticação forte para aplicações cliente/servidor.

\subsection{Arquitetura e Componentes}
O ecossistema Kerberos é composto por três entidades lógicas principais:
\begin{itemize}
    \item \textbf{Cliente:} A entidade (usuário ou software) que deseja acessar um recurso.
    \item \textbf{Servidor de Aplicação (SS - Service Server):} O recurso que o cliente deseja acessar (ex: servidor de arquivos).
    \item \textbf{KDC (Key Distribution Center):} A terceira parte confiável. O KDC mantém um banco de dados com as chaves secretas de todos os usuários e serviços. Logicamente, ele é dividido em dois subcomponentes:
    \begin{itemize}
        \item \textit{Authentication Server (AS):} Responsável pelo login inicial e emissão do TGT.
        \item \textit{Ticket Granting Server (TGS):} Responsável por emitir tickets para serviços específicos com base em um TGT válido.
    \end{itemize}
\end{itemize}

\subsection{Funcionamento e Troca de Mensagens}
O fluxo de autenticação no Kerberos ocorre em etapas distintas, desenhadas para garantir que a senha do usuário nunca trafegue pela rede:

\begin{enumerate}
    \item \textbf{Solicitação de Autenticação (AS\_REQ):} O cliente envia uma solicitação ao AS informando sua identidade (em texto claro).
    \item \textbf{Emissão do TGT (AS\_REP):} O AS verifica se o usuário existe. Se sim, gera uma chave de sessão e um Ticket de Concessão de Tickets (TGT). O TGT é criptografado com a chave do TGS, e a resposta para o cliente é criptografada com a chave derivada da senha do usuário. O cliente decifra essa resposta, obtendo a chave de sessão e o TGT, sem que a senha tenha saído de sua máquina.
    \item \textbf{Solicitação de Serviço (TGS\_REQ):} Quando o cliente precisa acessar um recurso (ex: impressora), ele envia ao TGS o seu TGT e um "Autenticador".
    \item \textbf{Emissão do Ticket de Serviço (TGS\_REP):} O TGS valida o TGT. Se válido, gera um Ticket de Serviço (criptografado com a chave do servidor de destino) e o envia ao cliente.
    \item \textbf{Acesso ao Recurso (AP\_REQ):} O cliente apresenta o Ticket de Serviço ao servidor da aplicação, que o decifra e valida a identidade do cliente, permitindo o acesso.
\end{enumerate}

% Subseção: Vulnerabilidades e Mitigações - SAMICO 
\subsection{Mitigação de Ataques}
\textcolor{red}{[SAMICO]}

Uma das principais ameaças em autenticação distribuída é o "Ataque de Repetição" (\textit{Replay Attack}), onde um atacante intercepta um ticket válido e o reenvia para o servidor para ganhar acesso não autorizado.

O Kerberos mitiga isso através do uso de \textit{Timestamps} (carimbos de tempo). Cada ticket e autenticador possui a hora de criação e um tempo de vida (TTL) curto (geralmente 8 a 10 horas para TGTs e minutos para autenticadores). Se um servidor receber um pacote com um horário muito diferente do seu relógio local (fora de uma janela de tolerância, comumente 5 minutos), a solicitação é rejeitada. Isso implica que a sincronização de relógios (via NTP) é um requisito obrigatório para o funcionamento de redes Kerberos.
