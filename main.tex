\documentclass[12pt]{article}

% Importação do estilo SBC movida para a pasta 'styles/' para organizar a raiz.
\usepackage{styles/sbc-template}

% Configurações adicionadas para garantir fidelidade ao template original:
% [T1]{fontenc} melhora a renderização de caracteres acentuados e hifenização.
% {mathptmx} força o uso da fonte Times (NimbusRom), essencial para manter o 
% layout, espaçamento e visual idênticos ao PDF de referência da SBC.
\usepackage[T1]{fontenc}
\usepackage{mathptmx}
\usepackage{graphicx,url}
\usepackage[utf8]{inputenc}
\usepackage[brazil]{babel}
% O uso de latin1 (ISO-8859-1) foi mantido comentado por ser um padrão antigo.
% O projeto agora utiliza UTF-8 (\usepackage[utf8]{inputenc}), que é o padrão 
% moderno e universal para caracteres acentuados.
%\usepackage[latin1]{inputenc}  

% Define o caminho das imagens para a pasta 'images/', permitindo manter a raiz limpa.
\graphicspath{{images/}}
     
\sloppy

\title{Artigo final de sistemas distribuídos - Análise do Protocolo Kerberos: Autenticação e Segurança em Sistemas Distribuídos}

\author{Luiz Antônio Lima de Freitas Leite\inst{1}, Max José Lobato Pantoja Junior\inst{1}, \\Wesley Pontes Barbosa\inst{1}, Luiz Sérgio Samico Maciel\inst{1}}


\address{Instituto de Ciências Exatas e Naturais (ICEN) -- Universidade Federal Pará\\ Belém, PA -- Brasil
\email{\{luiz.freitas.leite,max.junior,wesley.pontes.barbosa,luiz.filho\}@icen.ufpa.br}
}

% Título e Autores (Baseado na Lista da Equipe 1)
\title{}

\begin{document} 

\maketitle

% RESUMO
\begin{abstract}
  This article presents a study on digital security techniques in distributed systems, focusing on the Kerberos protocol. It explores the concepts of authentication, trusted third parties, and symmetric cryptography. Furthermore, a case study on its implementation in directory services (Active Directory) is presented, demonstrating its effectiveness in mitigating network threats.
\end{abstract}

\begin{resumo}
  Este artigo apresenta um estudo sobre técnicas de segurança digital em sistemas distribuídos, com foco no protocolo Kerberos. São explorados os conceitos de autenticação, terceira parte confiável e criptografia simétrica. Além disso, é apresentado um estudo de caso sobre sua implementação em serviços de diretório (Active Directory), demonstrando sua eficácia na mitigação de ameaças em rede.
\end{resumo}


% SEÇÃO 1: INTRODUÇÃO
\section{Introdução}
% O QUE ESCREVER AQUI:
% 1. Contextualize: Em Sistemas Distribuídos, a comunicação ocorre em redes inseguras.
% 2. O Problema: Como provar quem você é (autenticação) para um servidor remoto sem enviar sua senha pela rede, onde ela poderia ser interceptada (sniffing)?
% 3. A Solução proposta: O uso de criptografia e protocolos de autenticação robustos, especificamente o Kerberos.
% 4. Objetivo do trabalho: Analisar a arquitetura do Kerberos e sua aplicação prática.

A segurança é um dos desafios fundamentais no projeto de Sistemas Distribuídos. Diferente de sistemas centralizados, onde o sistema operacional tem controle total sobre o acesso à memória e recursos, sistemas distribuídos dependem de redes de comunicação que podem ser acessíveis a terceiros mal-intencionados...


% SEÇÃO 2: FUNDAMENTAÇÃO TEÓRICA
\section{Conceitos de Segurança em Sistemas Distribuídos}
% O QUE ESCREVER AQUI:
% Fale brevemente sobre os pilares que sustentam o Kerberos antes de explicar o protocolo em si.

\subsection{Canais Seguros e Criptografia}
% Explique a diferença entre Criptografia Simétrica (chave única - muito usada no Kerberos por ser rápida) e Assimétrica.
Para garantir a confidencialidade e integridade dos dados trafegados...

\subsection{Autenticação e Terceira Parte Confiável (TTP)}
% Explique o conceito de TTP (Trusted Third Party). O Kerberos funciona porque todo mundo confia no servidor central (KDC).
A autenticação em larga escala torna-se inviável se cada servidor precisar armazenar as senhas de todos os usuários. Surge então a necessidade de uma autoridade central confiável...


% SEÇÃO 3: O PROTOCOLO KERBEROS (O CORAÇÃO DO ARTIGO)
\section{O Protocolo Kerberos}
% O QUE ESCREVER AQUI:
% Esta é a parte técnica pesada. Usem diagramas se possível.

\subsection{Arquitetura e Componentes}
% Descreva os componentes:
% - Cliente
% - Servidor de Aplicação (SS)
% - KDC (Key Distribution Center), que se divide em:
%     - AS (Authentication Server)
%     - TGS (Ticket Granting Server)

O Kerberos, desenvolvido pelo MIT, baseia-se no modelo de chave simétrica de Needham-Schroeder. Sua arquitetura é composta por três entidades principais...

\subsection{Funcionamento e Troca de Mensagens}
% Descreva o fluxo básico (pode ser em texto ou itens):
% 1. Login inicial (Cliente pede TGT ao AS).
% 2. Cliente recebe o TGT (Ticket Granting Ticket).
% 3. Cliente pede acesso a um serviço (ex: Impressora) apresentando o TGT ao TGS.
% 4. TGS dá um Ticket de Serviço.
% 5. Cliente acessa o serviço.

Uma característica fundamental do Kerberos é o uso de "Tickets". O usuário não se autentica em cada serviço; ele se autentica uma vez no AS e recebe um bilhete mestre (TGT)...

\subsection{Mitigação de Ataques}
% Explique como ele evita "Replay Attacks" usando Carimbos de Tempo (Timestamps).
Para evitar que um atacante copie um ticket válido e o reutilize posteriormente (ataque de repetição), o Kerberos utiliza rigorosos carimbos de tempo...


% SEÇÃO 4: ESTUDO DE CASO (POSSÍVEIS APLICAÇÕES)
\section{Estudo de Caso: Microsoft Active Directory}
% O QUE ESCREVER AQUI:
% O professor pediu "Possíveis Aplicações". O Active Directory (AD) é a maior implementação de Kerberos do mundo.

Embora o Kerberos tenha nascido no mundo Unix/Linux (projeto Athena do MIT), sua adoção massiva ocorreu através do Microsoft Active Directory (AD). Em um ambiente corporativo Windows...

\subsection{Single Sign-On (SSO)}
% Explique a vantagem prática: O usuário loga no Windows de manhã e acessa e-mail, pastas compartilhadas e impressoras sem digitar a senha de novo. Isso é o Kerberos trabalhando em background.
A aplicação prática mais visível do Kerberos é a capacidade de \textit{Single Sign-On}. O usuário insere suas credenciais apenas na estação de trabalho...


% SEÇÃO 5: CONCLUSÃO
\section{Conclusão}
% Resuma o que foi aprendido.
% Ponto crítico: O Kerberos é excelente, mas o KDC é um "Ponto Único de Falha" (se o servidor de senhas cai, ninguém acessa nada).
O estudo do protocolo Kerberos demonstra a importância de mecanismos centralizados de confiança em ambientes distribuídos. Apesar de sua robustez, a centralização no KDC exige estratégias de replicação para evitar indisponibilidade...

\bibliographystyle{sbc}
\begin{thebibliography}{}

% EXAMPLES DE BIBLIOGRAFIA - Substituam pelos que vocês usarem
\bibitem[Coulouris et al. 2013]{coulouris}
Coulouris, G., Dollimore, J., Kindberg, T., and Blair, G. (2013).
\newblock {\em Distributed Systems: Concepts and Design}.
\newblock Pearson, 5th edition.

\bibitem[Tanenbaum and van Steen 2017]{tanenbaum}
Tanenbaum, A.~S. and van Steen, M. (2017).
\newblock {\em Distributed Systems: Principles and Paradigms}.
\newblock Pearson Education, 3rd edition.

\bibitem[Neuman and Ts'o 1994]{neuman}
Neuman, B.~C. and Ts'o, T. (1994).
\newblock Kerberos: An authentication service for computer networks.
\newblock {\em IEEE Communications Magazine}, 32(9):33--38.

\end{thebibliography}

\end{document}
Dicas para preencher o conteúdo:

Imagens: O professor adora diagramas. Tentem criar (ou pegar uma livre de direitos) uma imagem que mostre o "bonequinho" falando com o KDC, pegando o Ticket e indo pro Servidor. No LaTeX, usem \begin{figure}.

Foco nos Termos de SD: Sempre que possível, usem termos como "Cliente-Servidor", "Concorrência" (no acesso ao KDC), "Sincronização" (o relógio do cliente e servidor tem que estar igual, senão o Kerberos falha).

Tamanho: O limite é 10 páginas, mas um artigo bom de 4 a 6 páginas é suficiente para tirar nota máxima se o conteúdo for denso e bem escrito.

% Estilo de citação da SBC agora localizado na pasta 'bib/'.
\bibliographystyle{bib/sbc}
% Arquivo de dados bibliográficos (.bib) agora localizado na pasta 'bib/'.
\bibliography{bib/sbc-template}

\end{document}
